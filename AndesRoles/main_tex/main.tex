%! TeX program = pdflatex
\documentclass{article}
\usepackage[english]{babel}
\usepackage[utf8]{inputenc}
\usepackage{amsmath}
\usepackage{amssymb}
\usepackage{graphicx}
% \usepackage{natbib}
\usepackage{pgf}
% \usepackage{url}
\usepackage{algorithm}% http://ctan.org/pkg/algorithms
\usepackage{algpseudocode}% http://ctan.org/pkg/algorithmicx
% \usepackage{tikz-cd}
\usepackage{listings}
% \usepackage{xcolor}
% \usepackage[colorlinks, linkcolor = blue, citecolor = magenta]{hyperref}
\usepackage{tikz}
% % \usepackage{minted}
\usepackage{float}
% \usetikzlibrary{shapes, arrows, positioning}
% \lstset{ 
%     language=Python,                 % the language of the code
%     basicstyle=\ttfamily\small,      % the size of the fonts that are used for the code
%     numbers=left,                    % where to put the line-numbers
%     numberstyle=\tiny\color{gray},   % the style that is used for the line-numbers
%     stepnumber=1,                    % the step between two line-numbers. If it's 1, each line will be numbered
%     numbersep=5pt,                   % how far the line-numbers are from the code
%     backgroundcolor=\color{white},   % choose the background color. You must add \usepackage{color}
%     showspaces=false,                % show spaces adding particular underscores
%     showstringspaces=false,          % underline spaces within strings
%     showtabs=false,                  % show tabs within strings adding particular underscores
%     frame=single,                    % adds a frame around the code
%     rulecolor=\color{black},         % if not rolframe-color may be changed on line-breaks within not black text (e.g. comments (green here))
%     tabsize=4,                       % sets default tabsize to 4 spaces
%     captionpos=b,                    % sets the caption-position to bottom
%     breaklines=true,                 % sets automatic line breaking
%     breakatwhitespace=false,         % sets if automatic breaks should only happen at whitespace
%     title=\lstname,                  % show the filename of files included with \lstinputlisting; also try caption instead of title
%     keywordstyle=\color{blue},       % keyword style
%     commentstyle=\color{green},      % comment style
%     stringstyle=\color{red},         % string literal style
%     escapeinside={\%*}{*)},          % if you want to add LaTeX within your code
%     morekeywords={*,...} 
% }            % if you want to add more keywords to the rol\newenvironment{notation}
\newtheorem{remark}{Remark}
\newtheorem{definition}{Definition}
\newtheorem{property}{Property}

\newcommand{\pder}[2]{\frac{\partial #1}{\partial #2}}

\title{Electrical grid simulation in the COLMENA framework}
\author{Pablo de Juan Vela $^{1}$ \\
        \small $^{1}$eRoots, Barcelona, Spain \\
}
\date{\today}

\usepackage[style=ieee]{biblatex}
\addbibresource{ref.bib}
\setlength{\parskip}{1em} 
\begin{document}
\maketitle

\section{Introduction}
% ANDES intro
ANDES Curent \cite{grids:models} is a python package that can be used to model, simulate, and analyze power systems. The package uses predefined models that simulate different elements of an electrical grid. Additionally, the package allows the user to define custom models which relative ease. ANDES stands out from other simulation tools for the use of a hybrid symbolic-numeric framework for modeling differential algebraic equations (DAEs). This project present the objectives for developing test case for the COLMENA framework in the context of electrical grids using the simulation tools provided by ANDES. The goal of the project is to showcase the capabilities of COLMENA in the context of a decentralized electrical grid. We will outline the different requirements needed 

\subsection{Power Grid \& Service definition}
We define an electrical grid as a set of nodes called buses with electrical devices attached to them, the nodes are interconnected with lines. These devices include generators, loads or others types of devices. In ANDES, each device is defined by a set of differential and algebraic equations (DAE) that define how the devices' states change over time.
\vspace{1em} 
The electrical grid is defined by multiple metrics that define the state of the grid. Some of the more critical metrics that we consider are the following.

\begin{itemize}
    \item Frequency: The frequency is a local metric that describes the voltage at a specific bus. A device called Phasor Measurement Unit (PMU) can measure the frequency of the bus it is connected to. Maintaining the frequency at the nominal value of $50 Hz$ is key for the proper functioning of the grid. Values of the frequency above or below this value can unbalance the grid from a power point of view or even damage certain components.   
    \item Synchronous Rotor's Speed: Synchronous generators are device that generate power and then inject to the grid. This is done by spinning a rotor that spins at a certain speed, ideally as close as to the nominal frequency. The value of the frequency in the grids buses and the angular speed of the rotor are closely linked and are frequently used interchangeably. In fact, generators can see a drop in their angular speed to mitigate a drop in frequency in a close bus.
    \item Voltage $(V)$: The voltage is a metric linked to a specific bus and also needs to be as close to the nominal value as possible. Drops in frequency can be linked to drops in voltage.
    \item Power injected/consumed $(kW)$: The power injected is by a generator is the amount of power the . The objective of the grid supervisor is to keep the balance of power as close to zero as possible. Raising the amount of power injected by a generator also tends to raise its synchronous speed. It's an important lever in controlling  
\end{itemize}

The metrics just introduced are key to defining the performance of a grid and to the frequency control response of a power grid. The objective of the frequency response is multiple. On a first time scale (primary response), the control aims to take the system from a transient to a stable condition and avoiding critical values for the frequency. During the secondary response, the objective is to restore the steady state values to their nominal values. Finally, the last response's objective is to restore the the power reserves to their original value.
\vspace{1em} 
The objective of the service implemented in COLMENA will be then to aid to the frequency control response by implementing decentralized roles that can improve the performance of the control. This roles will take the form of changing set-points for certain devices, connecting or disconnecting devices such as secondary generators and others. 

\subsection{ANDES' Devices}

A simulation created by the ANDES packages is organized around devices. Each device at a given time $t$ is defined by the value's of the algebraic variables and state variables. Devices of the same type have the same variables. Buses are one of the building blocks of the grid modeling. Other type of devices will connect to these buses. In the context of the frequency service control we have found the following devices that can be useful to the simulation and also present some sort of decentralized control.

\begin{itemize}
    \item Synchronous Generators: They inject power to the grid.
    \item Converters \& distributed generations: Used to convert power from distributed generation from DC to AC, usually associated to some sort of distributed generation.
    \item Loads: They consume power from the grid.
    \item Lines: They connect different buses.
    \item Switches: They allow the flow of power through a Line.
\end{itemize}

In the following sections we will se how these devices fit with the service of controlling the frequency.

\subsubsection*{Synchronous Generators}

Generators are devices that inject power to the grid. A synchronous generator injects the power through a rotating part that spins synchronously to the grid's frequency. The synchronous generators in the simulation will be defined by two internal states: $\omega \in \mathbb{R}$ the angular velocity and $\theta  \in \mathbb{R}$. The generator's turbine can set the value of the power being generator's or the set point for the reference angular speed $\omega_{ref}$. Controlling these set points is key to improve the performance of the grid.

\subsubsection*{Converters \& distributed generation}

The modelling of distributed generation in ANDES usually combines the generation considered as a DC current source and a battery that is connected to a AC-DC converter. The converter that is paired to this ensemble can control both the active and reactive power that is injected to the grid and that is stored in the battery (if present). The control of the setup is defined by the parameters $\gamma_p, \gamma_q \in [0,1]$ . These parameters define which proportion of the active and reactive power respectively and generated by the setup is injected to the grid. We can therefore define different operating points for the distributed generation depending on the power injected. The power that is not injected is then saved by the battery. This can be expressed as these three different operating points:

\begin{itemize}
    \item $\gamma_p = 1$ all of the power generated is injected to the grid.
    \item $\gamma_p = 0$ all of the power generated is stored in the battery.
    \item $\gamma_p = 0.5$ half of the power generated is injected to the grid and half is stored in the battery.
\end{itemize}


In the setup that we just explained the converter can take the form of a Voltage Source Converter (VSC). A VSC is a converter (converts a DC current to AC) with some differences to classical converters. They are commonly used to track the angle of the grid at a bus and inject (or absorb) active and reactive power from an energy source. They are usually paired with DC sources of energy as a way to connect them to the grid. We consider a VSC that can operate in two modes: 'Grid following'(GFL) and 'Grid forming'(GFM). 

\begin{itemize}
    \item Grid Following: In the GFL mode, the converter tracks the grid's frequency and injects a controlled power, acting as an ideal current source.
    \item Grid Forming: In the GFM mode, the converter creates its own frequency and imposes a voltage differential acting as an ideal voltage source.
\end{itemize}

In terms of control, both GFM and GFL have a reference value with a control. The control of the devices aims to get as close to the reference value as possible. For the GFL mode the reference value usually refers to the power injected, while for the GFM its the frequency or voltage. This GFM behavior is quite close to one of a typical synchronous generator. The flexibility and the different set points of this type of converter is fits well with the decentralized control principle of COLMENA.  

\subsubsection*{Loads}

A load is an electrical model that is connected to a bus and consumes a given amount of active and reactive($P(MW),  Q (Mvar)$ respectively). In the context of the frequency control service, the load device can have varying values of $P$ and $Q$ depending on the state of the grid. More specifically, . These adaptable response can be very useful to the grid in order to adapt to drops in generation or line faults.

\subsubsection*{Switches}
Switches are controllable elements that control the connection state of a Line. In this case the operating states are just 'Open' or 'Closed'. In the open state no current travels directly between the buses while in the Closed state the line works at normal operation. \\

\subsection{Andes-COLMENA Integration requirements}

In this section we explain how Andes and COLMENA will work during the simulation and how they will communicate. As a standalone, ANDES simulates the time domain response of the power grid for a pre-set amount of time. The package solves the equations numerically and return the solution but it doesn't simulate the grid in real time, instead it solves it as fast as possible. To enable the integration with COLMENA we modify ANDES into running for a given step-size only when the real time is greater than the simulation time in order for it to catch-up to. ANDES is run parallel to COLMENA on an independent app run on an independent device not part of COLMENA's colony of agents. Specific COLMENA Agents will periodically send requests to the ANDES App to get the simulation's information and to send changes   

From COLMENA's side, we first consider the case were each COLMENA-Agent is paired with a specific ANDES-device.  

\section{Use Case Specification}

\subsection{Grid Definition}

\subsection{Grid's Performance \& Key Performance Indicators}

In the first section we have defined a set of metrics and states that are key to defining the grids state, specifically when dealing with the frequency response. These include the frequency, the angular speed of the synchronous generators and others. We aim to define key performance indicators (KPI) starting from the metrics defined previously. These KPIs define the performance of the grid with respect to the frequency service control. 

\begin{itemize}
    \item Deviation of the generator's frequency.
    \item Deviation of the mean bus frequency in an area from the nominal frequency.
    \item Steady state frequency value.
    \item Rate of Change of Frequency (RoCoF).
    \item Area Control Error (ACE).
\end{itemize}

\nocite{*}
\printbibliography

\end{document}