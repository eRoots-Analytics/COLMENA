%! TeX program = pdflatex
\documentclass{article}
\usepackage[english]{babel}
\usepackage[utf8]{inputenc}
\usepackage{amsmath}
\usepackage{amssymb}
\usepackage{minted}
\usepackage{listings}
\usepackage{xcolor}
\usepackage{graphicx}
\usepackage{pgf}
\usepackage{algorithm}% http://ctan.org/pkg/algorithms
\usepackage{algorithmicx}
\usepackage{algpseudocode}% http://ctan.org/pkg/algorithmicx
% \usepackage{tikz-cd}
\usepackage{listings}
\usepackage{xcolor}
\usepackage{tikz}
\usepackage{float}
% \usetikzlibrary{shapes, arrows, positioning}
% \lstset{ 
%     language=Python,                 % the language of the code
%     basicstyle=\ttfamily\small,      % the size of the fonts that are used for the code
%     numbers=left,                    % where to put the line-numbers
%     numberstyle=\tiny\color{gray},   % the style that is used for the line-numbers
%     stepnumber=1,                    % the step between two line-numbers. If it's 1, each line will be numbered
%     numbersep=5pt,                   % how far the line-numbers are from the code
%     backgroundcolor=\color{white},   % choose the background color. You must add \usepackage{color}
%     showspaces=false,                % show spaces adding particular underscores
%     showstringspaces=false,          % underline spaces within strings
%     showtabs=false,                  % show tabs within strings adding particular underscores
%     frame=single,                    % adds a frame around the code
%     rulecolor=\color{black},         % if not rolframe-color may be changed on line-breaks within not black text (e.g. comments (green here))
%     tabsize=4,                       % sets default tabsize to 4 spaces
%     captionpos=b,                    % sets the caption-position to bottom
%     breaklines=true,                 % sets automatic line breaking
%     breakatwhitespace=false,         % sets if automatic breaks should only happen at whitespace
%     title=\lstname,                  % show the filename of files included with \lstinputlisting; also try caption instead of title
%     keywordstyle=\color{blue},       % keyword style
%     commentstyle=\color{green},      % comment style
%     stringstyle=\color{red},         % string literal style
%     escapeinside={\%*}{*)},          % if you want to add LaTeX within your code
%     morekeywords={*,...} 
% }            % if you want to add more keywords to the rol\newenvironment{notation}
\newtheorem{remark}{Remark}
\newtheorem{definition}{Definition}
\newtheorem{property}{Property}

\newcommand{\pder}[2]{\frac{\partial #1}{\partial #2}}

\title{Electrical grid simulation in the COLMENA framework}
\author{Pablo de Juan Vela $^{1}$ \\
        \small $^{1}$eRoots, Barcelona, Spain \\
}
\date{\today}

\usepackage[style=ieee]{biblatex}
\addbibresource{ref.bib}
\setlength{\parskip}{1em} 
\begin{document}
\maketitle

\section{MPC Formulation}
The decision variables are:
\begin{itemize}
    \item $f_t$ the area's frequency.
    \item $\delta_{i,t}$ the angle of area i.
    \item $P^{\text{gen}_i}_t$ the generator's i power output in pu.
    \item $P^{i,j}_t$ the power exchanged between area i and area j in pu.
\end{itemize}

The parameters are:
\begin{itemize}
    \item $B_{i,j}$ the susceptance between area i and area j in p.u.
    \item $\hat{\delta}_{i,j,t}$ is the previous optimal solution of the MPC of area i for the area angles $\delta_{j,t}^*$.
    \item $\hat{\delta}_{i,0}$ the initial value of the area i in radians.
    \item $\hat{f_{0}}$ the current area initial frequency value.
    \item $u^{min}_{gen_i}, u^{max}_{gen_i}$ the minimum and maximum ramp up speed of generator i in pu/s.
    \item $M_i$ the area wise inertia constant defined as $\frac{\sum_{gen_k \in AreaGenerators} Sn^{gen_k}M_{gen_k}}{\sum_{gen_k \in AreaGenerators} Sn^{gen_k}}$ 
    \item $D_i$ the area wise damping coefficient defined as $\frac{\sum_{gen_k \in AreaGenerators} Sn^{gen_k}D_{gen_k}}{\sum_{gen_k \in AreaGenerators} Sn^{gen_k}}$ 
\end{itemize}


The dynamics of the MPC model can be expressed as the following differential equations:

\begin{align}
    \dot{\delta}_i &= 2 \pi (f_t - f_0) \quad  \forall t = 0, \dots, T\\
    M(\dot{f}) &= -D(f-f_0) + \sum_{gen \in area} P^{gen}_t +\sum_{j \in \text{other area}}P_{i,j} - P^{demand}_t 
\end{align}
The MPC formulation for area $i$ is.
\begin{align}
    \min_{f_{t}, \delta_{i,t}, P^{gen}_{t}, P^{i,j}_t} \quad 
    & \|f - f_0\|^2 
    + \sum_{j \in \text{NeighboringAreas}} \lambda_{j, i, t} (\delta_{j,t} - \hat{\delta}_{i,j,t}) \cdot \text{sign}(i-j) \\
    & + \sum_{j \in \text{NeighboringAreas}} \lambda_{i,j,t} (\delta_{i,t} - \hat{\delta}_{j,i,t}) \cdot \text{sign}(j-i) \nonumber \\
    & + \sum_{j \in \text{NeighboringAreas}} (\delta_{j,t} - \hat{\delta}_{i,j,t})^2 \nonumber \\
    \text{s.t.} \quad 
    & \delta_{t+1} -\delta_{t} = 2 \Delta t \pi (f_t - f_0) \quad \forall t = 0, \dots, T \\
    & M(f_{t+1} - f_t) = -\Delta t \left(D(f-f_0) + \sum_{\text{gen} \in \text{area}} P^{gen}_t + \sum_{j \in \text{other area}} P_{i,j} - P^{demand}_t \right) \\
    & \delta_{j,0} = \hat{\delta}_{j,0}  \forall j \in \text{Areas}\\
    & f_0 = \hat{f}_0 \\
    & P_{i,j} = \sum_{j \in \text{Areas}} B_{i,j}(\delta_{i,t} - \delta_{j,t}) \\
    & u_{\min} \leq P^{gen_k }_{t+1} - P^{gen_k }_{t} \leq u_{\max} \quad \forall gen_k \in AreaGenerators \forall t = 0,...,T-1\\
    & P^{gen_k}_{\min} \leq P^{gen}_{i, t} \leq P^{gen_k}_{\max} \quad \forall gen_k \in AreaGenerators
\end{align}

Where we have that this expression is the cost in the objective function associated to the error of the agent's data about other agents angles,
\begin{equation}
    \sum_{j \in NeighboringAreas}\lambda_{j, i, t}(\delta_{j,t} - \hat{\delta}_{i,j,t})(sign(i-j))
\end{equation} 
whereas this other expression is the error of the other agent's data tells about the value they have.
\begin{equation}
    \sum_{j \in NeighboringAreas}\lambda_{i,j,t}(\delta_{i,t} - \hat{\delta}_{j,i,t})(sign(j-i))
\end{equation} 

The objective of the extended formulation is to expand the current problem where we consider the areas as a simple aggregation of devices to one where the topology inside each grid is also considered. For this purpose, we add new constraints related to the power balance in each individual bus and not just in the whole area.
 
\begin{align}
    P^{gen}_{\text{bus}_i} - P^d_{\text{bus}_i} - \sum_{j \in area} P_{\text{bus}_i, \text{bus}_j}= 0 \quad \forall \text{bus}_i \in \text{Area}\\
    P_{\text{bus}_i, \text{bus}_j} = B_{i,j}(\theta_{i}-\theta_{j})
\end{align}
\begin{algorithm}
    \caption{Distributed MPC ADMM algorithm}
    \label{algo:ADMM}
    \begin{algorithmic}[1]
        \State Initialize grid states $x$ to $x_0$ using Power Flow results.
        \While{error $>$ tolerance}
            \For{each agent in area\_agents}
                \State Agent controlling area i
                \State Get initial values $f_0^{\text{area}_i}$, $\delta_{\text{area}, 0} \quad \forall \text{agent}$ from other agents via \texttt{@Data(horizon)}
                \State Get state horizon of the other $\hat{\delta}_{\text{area}, 0} \quad \forall \text{agent}$ from other agents via \texttt{@Data(horizon)}
                \State Solve local area MPC
                \State Publish state horizon solution $\delta_{\text{area}_i, t}^* \forall t=0...T$ to \texttt{@Data(horizon)}
            \EndFor
            \State $error \gets \delta_{i}^* - \hat{\delta_{i}}$
            \State $\lambda_{i} \gets \lambda_{i} + \alpha \cdot error$
        \EndWhile
        \For{gen in generators}
            \State Consider the area's MPC the generator belongs to.
            \State Update generator power setpoint $P_{ref, gen} \gets (P^{g}_{gen, 0})^*$
        \EndFor
    \end{algorithmic}
\end{algorithm}

\subsection{Distributed MPC with role dynamics}

\nocite{*}
\printbibliography
\end{document}