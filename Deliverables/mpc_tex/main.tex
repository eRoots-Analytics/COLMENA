%! TeX program = pdflatex
\documentclass{article}
\usepackage[english]{babel}
\usepackage[utf8]{inputenc}
\usepackage{amsmath}
\usepackage{amssymb}
\usepackage{minted}
\usepackage{listings}
\usepackage{xcolor}
\usepackage{graphicx}
\usepackage{pgf}
\usepackage{algorithm}% http://ctan.org/pkg/algorithms
\usepackage{algorithmicx}
\usepackage{algpseudocode}% http://ctan.org/pkg/algorithmicx
% \usepackage{tikz-cd}
\usepackage{listings}
\usepackage{xcolor}
\usepackage{tikz}
\usepackage{float}
% \usetikzlibrary{shapes, arrows, positioning}
% \lstset{ 
%     language=Python,                 % the language of the code
%     basicstyle=\ttfamily\small,      % the size of the fonts that are used for the code
%     numbers=left,                    % where to put the line-numbers
%     numberstyle=\tiny\color{gray},   % the style that is used for the line-numbers
%     stepnumber=1,                    % the step between two line-numbers. If it's 1, each line will be numbered
%     numbersep=5pt,                   % how far the line-numbers are from the code
%     backgroundcolor=\color{white},   % choose the background color. You must add \usepackage{color}
%     showspaces=false,                % show spaces adding particular underscores
%     showstringspaces=false,          % underline spaces within strings
%     showtabs=false,                  % show tabs within strings adding particular underscores
%     frame=single,                    % adds a frame around the code
%     rulecolor=\color{black},         % if not rolframe-color may be changed on line-breaks within not black text (e.g. comments (green here))
%     tabsize=4,                       % sets default tabsize to 4 spaces
%     captionpos=b,                    % sets the caption-position to bottom
%     breaklines=true,                 % sets automatic line breaking
%     breakatwhitespace=false,         % sets if automatic breaks should only happen at whitespace
%     title=\lstname,                  % show the filename of files included with \lstinputlisting; also try caption instead of title
%     keywordstyle=\color{blue},       % keyword style
%     commentstyle=\color{green},      % comment style
%     stringstyle=\color{red},         % string literal style
%     escapeinside={\%*}{*)},          % if you want to add LaTeX within your code
%     morekeywords={*,...} 
% }            % if you want to add more keywords to the rol\newenvironment{notation}
\newtheorem{remark}{Remark}
\newtheorem{definition}{Definition}
\newtheorem{property}{Property}

\newcommand{\pder}[2]{\frac{\partial #1}{\partial #2}}

\title{Electrical grid simulation in the COLMENA framework}
\author{Pablo de Juan Vela $^{1}$ \\
        \small $^{1}$eRoots, Barcelona, Spain \\
}
\date{\today}

\usepackage[style=ieee]{biblatex}
\addbibresource{ref.bib}
\setlength{\parskip}{1em} 
\begin{document}
\maketitle

\section{MPC Formulation}
The decision variables are:
\begin{itemize}
    \item $f_t$ the area's frequency.
    \item $\delta_{i,t}$ the area's angle.
    \item $P^{gen}_t$ the generator power output.
    \item $P^{i,j}_t$ the power exchanged between area i and area j.
\end{itemize}

The parameters are:
\begin{itemize}
    \item $B_{i,j}$ the susceptance between area i and arera j.
    \item $\hat{\delta_{i,0}}$ the area's angle initial values.
    \item $\hat{f_{0}}$ the area's frequency initial values.
\end{itemize}

The MPC formulation is
\begin{align}
    \min_{f_{t}, \delta_{i,t}, P_^{gen}{t}, P^{i,j}_t} &||f - f_0||^2 + \sum_{j \in OtherAreas} \lambda_j(\delta_{j,t} - \hat{\delta}_{j,t}) + \sum_{j \in OtherAreas}(\delta_{j,t} - \hat{\delta}_{j,t})^2\\
    \text{s.t.} \quad  &\dot{\delta}_a = 2 \pi (f_t - f_0) \quad  \forall t = 0, \dots, T\\
    & M(\dot{f})= -D(f-f_0) + \sum_{gen \in area} P^{gen}_t +\sum_{j \in \text{other area}}P_{i,j} - P^{demand}_t \\
    & \delta_{i,0} = \hat{\delta}_{i,0} \\
    & f_0 = \hat{f}_0 \\
    & P_{i,j} = \sum_{j \in \text{Areas}} B_{i,j}(\delta_{i,t} - \delta_{j,t}) \\
    & u_{min} \leq P^{gen}_{i, t+1} - P^{gen}_{i, t} \leq u_{max} \\
    & P_{gen}^{min} \leq P^{gen}_{i, t} \leq P_{gen}^{max}
\end{align}

The extended formulation would add the following equation:

\begin{align}
    \sum_{j \in area} P_{\text{bus}_i, \text{bus}_j} + P^{gen}_{\text{bus}_i} +
\end{align}

\begin{algorithm}
    \caption{Distributed MPC ADMM algorithm}
    \label{algo:ADMM}
    \begin{algorithmic}[1]
        \State Initialize grid states $x$ to $x_0$ using Power Flow results.
        \While{error $>$ tolerance}
            \For{each agent in area\_agents}
                \State Agent controlling area i
                \State Get initial values $f_0^{\text{area}_i}$, $\delta_{\text{area}, 0} \quad \forall \text{agent}$ from other agents via \texttt{@Data(horizon)}
                \State Get state horizon of the other $\hat{\delta}_{\text{area}, 0} \quad \forall \text{agent}$ from other agents via \texttt{@Data(horizon)}
                \State Solve local area MPC
                \State Publish state horizon solution $\delta_{\text{area}_i, t}^* \forall t=0...T$ to \texttt{@Data(horizon)}
                \State $error \gets \delta_{i} - \hat{\delta_{i}}$
                \State $\lambda_{i} \gets \lambda_{i} + \alpha \cdot error$
            \EndFor
        \EndWhile
        \For{gen in generators}
            \State Consider the area's MPC the generator belongs to.
            \State Update generator power setpoint $P_{ref, gen} \gets (P^{g}_{gen, 0})^*$
        \EndFor
    \end{algorithmic}
\end{algorithm}

\subsection{Distributed MPC with role dynamics}

\nocite{*}
\printbibliography
\end{document}